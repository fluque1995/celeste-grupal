\documentclass[12pt]{article} 
\usepackage[utf8x]{inputenc}
\usepackage{graphicx} 
\usepackage{multirow} 
\usepackage{hhline}
\usepackage{booktabs} 
\usepackage{vmargin} %cambia el margen
\usepackage{amsmath,amsthm} 
\usepackage{amsfonts} 
\usepackage{float}
\usepackage{listings}
\usepackage{xcolor}

\usepackage{hyperref}

\hypersetup{
    bookmarks=true,         % show bookmarks bar?
    unicode=false,          % non-Latin characters in Acrobat’s bookmarks
    pdftoolbar=true,        % show Acrobat’s toolbar?
    pdfmenubar=true,        % show Acrobat’s menu?
    pdffitwindow=false,     % window fit to page when opened
    pdfstartview={FitH},    % fits the width of the page to the window
    pdftitle={Trabajo de mecánica celeste},    % title
    pdfauthor={Francisco Luque},     % author
    pdfsubject={Subject},   % subject of the document
    pdfcreator={Creator},   % creator of the document
    pdfproducer={Producer}, % producer of the document
    pdfkeywords={keyword1, key2, key3}, % list of keywords
    pdfnewwindow=true,      % links in new PDF window
    colorlinks=true,        % false: boxed links; true: colored links
    linkcolor=gray,         % color of internal links (change box color with linkbordercolor)
    citecolor=green,        % color of links to bibliography
    filecolor=magenta,      % color of file links
    urlcolor=blue           % color of external links
}

\title{
  Mecánica Celeste\\
  \large Trabajo grupal: Programación sobre la 
  órbita de los planetas en el sistema solar y
  resolución del problema de los dos cuerpos.  }


\author{ 
  Ana Isabel Gálvez Abad\\
  Francisco Luque Sánchez\\
  Lidia San Pedro Hernández
}

\begin{document}
\maketitle
\begin{center}  
\includegraphics[scale=0.35]{escudo.png}
\end{center}

\newpage

\section{Explicación de la ampliación de la práctica}

En esta práctica se ha ampliado el código dedarrollado en la práctica
individual para resolver el problema de los N cuerpos. Concretamente,
se ha resuelto este problema para ver cómo cambia la órbita de un
planeta cuando se considera la atracción que ejerce este sobre el Sol.\\

Además, se ha completado el código ya desarrollado para tener en cuenta
las desviaciones que tienen las órbitas de los planetas. Las elipses que
definen las órbitas de los planetas no se sitúan todas en el mismo plano,
por lo que se han creado dos gráficas en tres dimensiones en las que se
pueden apreciar estas diferencias. En una de las gráficas se muestran
los cuatro planetas interiores (Mercurio, Venus, La Tierra y Marte),
y en la otra los cuatro planetas exteriores (Júpiter, Saturno, Urano
y Neptuno).

\section{Ejecución de la práctica}

A fin de facilitar al usuario la ejecución de la práctica, se ha desarrollado
una aplicación interactiva, accesible a través de internet. La URL para
acceder a la aplicación es 
\url{https://mybinder.org/v2/gh/pacron/celeste-grupal/master?filepath=mecanica_grupal.ipynb}.
En dicha dirección se pueden ver de forma gráfica todos los resultados 
obtenidos.

\section{Código desarrollado}

Todo el código desarrollado para la aplicación está disponible a través
de internet, en la dirección \url{https://github.com/pacron/celeste-grupal}.
Además, en el archivo comprimido se adjunta también dicho código.

\end{document}